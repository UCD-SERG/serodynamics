% Options for packages loaded elsewhere
\PassOptionsToPackage{unicode}{hyperref}
\PassOptionsToPackage{hyphens}{url}
\PassOptionsToPackage{dvipsnames,svgnames,x11names}{xcolor}
%
\documentclass[
]{article}

\usepackage{amsmath,amssymb}
\usepackage{iftex}
\ifPDFTeX
  \usepackage[T1]{fontenc}
  \usepackage[utf8]{inputenc}
  \usepackage{textcomp} % provide euro and other symbols
\else % if luatex or xetex
  \usepackage{unicode-math}
  \defaultfontfeatures{Scale=MatchLowercase}
  \defaultfontfeatures[\rmfamily]{Ligatures=TeX,Scale=1}
\fi
\usepackage{lmodern}
\ifPDFTeX\else  
    % xetex/luatex font selection
\fi
% Use upquote if available, for straight quotes in verbatim environments
\IfFileExists{upquote.sty}{\usepackage{upquote}}{}
\IfFileExists{microtype.sty}{% use microtype if available
  \usepackage[]{microtype}
  \UseMicrotypeSet[protrusion]{basicmath} % disable protrusion for tt fonts
}{}
\makeatletter
\@ifundefined{KOMAClassName}{% if non-KOMA class
  \IfFileExists{parskip.sty}{%
    \usepackage{parskip}
  }{% else
    \setlength{\parindent}{0pt}
    \setlength{\parskip}{6pt plus 2pt minus 1pt}}
}{% if KOMA class
  \KOMAoptions{parskip=half}}
\makeatother
\usepackage{xcolor}
\setlength{\emergencystretch}{3em} % prevent overfull lines
\setcounter{secnumdepth}{5}
% Make \paragraph and \subparagraph free-standing
\makeatletter
\ifx\paragraph\undefined\else
  \let\oldparagraph\paragraph
  \renewcommand{\paragraph}{
    \@ifstar
      \xxxParagraphStar
      \xxxParagraphNoStar
  }
  \newcommand{\xxxParagraphStar}[1]{\oldparagraph*{#1}\mbox{}}
  \newcommand{\xxxParagraphNoStar}[1]{\oldparagraph{#1}\mbox{}}
\fi
\ifx\subparagraph\undefined\else
  \let\oldsubparagraph\subparagraph
  \renewcommand{\subparagraph}{
    \@ifstar
      \xxxSubParagraphStar
      \xxxSubParagraphNoStar
  }
  \newcommand{\xxxSubParagraphStar}[1]{\oldsubparagraph*{#1}\mbox{}}
  \newcommand{\xxxSubParagraphNoStar}[1]{\oldsubparagraph{#1}\mbox{}}
\fi
\makeatother

\usepackage{color}
\usepackage{fancyvrb}
\newcommand{\VerbBar}{|}
\newcommand{\VERB}{\Verb[commandchars=\\\{\}]}
\DefineVerbatimEnvironment{Highlighting}{Verbatim}{commandchars=\\\{\}}
% Add ',fontsize=\small' for more characters per line
\usepackage{framed}
\definecolor{shadecolor}{RGB}{241,243,245}
\newenvironment{Shaded}{\begin{snugshade}}{\end{snugshade}}
\newcommand{\AlertTok}[1]{\textcolor[rgb]{0.68,0.00,0.00}{#1}}
\newcommand{\AnnotationTok}[1]{\textcolor[rgb]{0.37,0.37,0.37}{#1}}
\newcommand{\AttributeTok}[1]{\textcolor[rgb]{0.40,0.45,0.13}{#1}}
\newcommand{\BaseNTok}[1]{\textcolor[rgb]{0.68,0.00,0.00}{#1}}
\newcommand{\BuiltInTok}[1]{\textcolor[rgb]{0.00,0.23,0.31}{#1}}
\newcommand{\CharTok}[1]{\textcolor[rgb]{0.13,0.47,0.30}{#1}}
\newcommand{\CommentTok}[1]{\textcolor[rgb]{0.37,0.37,0.37}{#1}}
\newcommand{\CommentVarTok}[1]{\textcolor[rgb]{0.37,0.37,0.37}{\textit{#1}}}
\newcommand{\ConstantTok}[1]{\textcolor[rgb]{0.56,0.35,0.01}{#1}}
\newcommand{\ControlFlowTok}[1]{\textcolor[rgb]{0.00,0.23,0.31}{\textbf{#1}}}
\newcommand{\DataTypeTok}[1]{\textcolor[rgb]{0.68,0.00,0.00}{#1}}
\newcommand{\DecValTok}[1]{\textcolor[rgb]{0.68,0.00,0.00}{#1}}
\newcommand{\DocumentationTok}[1]{\textcolor[rgb]{0.37,0.37,0.37}{\textit{#1}}}
\newcommand{\ErrorTok}[1]{\textcolor[rgb]{0.68,0.00,0.00}{#1}}
\newcommand{\ExtensionTok}[1]{\textcolor[rgb]{0.00,0.23,0.31}{#1}}
\newcommand{\FloatTok}[1]{\textcolor[rgb]{0.68,0.00,0.00}{#1}}
\newcommand{\FunctionTok}[1]{\textcolor[rgb]{0.28,0.35,0.67}{#1}}
\newcommand{\ImportTok}[1]{\textcolor[rgb]{0.00,0.46,0.62}{#1}}
\newcommand{\InformationTok}[1]{\textcolor[rgb]{0.37,0.37,0.37}{#1}}
\newcommand{\KeywordTok}[1]{\textcolor[rgb]{0.00,0.23,0.31}{\textbf{#1}}}
\newcommand{\NormalTok}[1]{\textcolor[rgb]{0.00,0.23,0.31}{#1}}
\newcommand{\OperatorTok}[1]{\textcolor[rgb]{0.37,0.37,0.37}{#1}}
\newcommand{\OtherTok}[1]{\textcolor[rgb]{0.00,0.23,0.31}{#1}}
\newcommand{\PreprocessorTok}[1]{\textcolor[rgb]{0.68,0.00,0.00}{#1}}
\newcommand{\RegionMarkerTok}[1]{\textcolor[rgb]{0.00,0.23,0.31}{#1}}
\newcommand{\SpecialCharTok}[1]{\textcolor[rgb]{0.37,0.37,0.37}{#1}}
\newcommand{\SpecialStringTok}[1]{\textcolor[rgb]{0.13,0.47,0.30}{#1}}
\newcommand{\StringTok}[1]{\textcolor[rgb]{0.13,0.47,0.30}{#1}}
\newcommand{\VariableTok}[1]{\textcolor[rgb]{0.07,0.07,0.07}{#1}}
\newcommand{\VerbatimStringTok}[1]{\textcolor[rgb]{0.13,0.47,0.30}{#1}}
\newcommand{\WarningTok}[1]{\textcolor[rgb]{0.37,0.37,0.37}{\textit{#1}}}

\providecommand{\tightlist}{%
  \setlength{\itemsep}{0pt}\setlength{\parskip}{0pt}}\usepackage{longtable,booktabs,array}
\usepackage{calc} % for calculating minipage widths
% Correct order of tables after \paragraph or \subparagraph
\usepackage{etoolbox}
\makeatletter
\patchcmd\longtable{\par}{\if@noskipsec\mbox{}\fi\par}{}{}
\makeatother
% Allow footnotes in longtable head/foot
\IfFileExists{footnotehyper.sty}{\usepackage{footnotehyper}}{\usepackage{footnote}}
\makesavenoteenv{longtable}
\usepackage{graphicx}
\makeatletter
\def\maxwidth{\ifdim\Gin@nat@width>\linewidth\linewidth\else\Gin@nat@width\fi}
\def\maxheight{\ifdim\Gin@nat@height>\textheight\textheight\else\Gin@nat@height\fi}
\makeatother
% Scale images if necessary, so that they will not overflow the page
% margins by default, and it is still possible to overwrite the defaults
% using explicit options in \includegraphics[width, height, ...]{}
\setkeys{Gin}{width=\maxwidth,height=\maxheight,keepaspectratio}
% Set default figure placement to htbp
\makeatletter
\def\fps@figure{htbp}
\makeatother
% definitions for citeproc citations
\NewDocumentCommand\citeproctext{}{}
\NewDocumentCommand\citeproc{mm}{%
  \begingroup\def\citeproctext{#2}\cite{#1}\endgroup}
\makeatletter
 % allow citations to break across lines
 \let\@cite@ofmt\@firstofone
 % avoid brackets around text for \cite:
 \def\@biblabel#1{}
 \def\@cite#1#2{{#1\if@tempswa , #2\fi}}
\makeatother
\newlength{\cslhangindent}
\setlength{\cslhangindent}{1.5em}
\newlength{\csllabelwidth}
\setlength{\csllabelwidth}{3em}
\newenvironment{CSLReferences}[2] % #1 hanging-indent, #2 entry-spacing
 {\begin{list}{}{%
  \setlength{\itemindent}{0pt}
  \setlength{\leftmargin}{0pt}
  \setlength{\parsep}{0pt}
  % turn on hanging indent if param 1 is 1
  \ifodd #1
   \setlength{\leftmargin}{\cslhangindent}
   \setlength{\itemindent}{-1\cslhangindent}
  \fi
  % set entry spacing
  \setlength{\itemsep}{#2\baselineskip}}}
 {\end{list}}
\usepackage{calc}
\newcommand{\CSLBlock}[1]{\hfill\break\parbox[t]{\linewidth}{\strut\ignorespaces#1\strut}}
\newcommand{\CSLLeftMargin}[1]{\parbox[t]{\csllabelwidth}{\strut#1\strut}}
\newcommand{\CSLRightInline}[1]{\parbox[t]{\linewidth - \csllabelwidth}{\strut#1\strut}}
\newcommand{\CSLIndent}[1]{\hspace{\cslhangindent}#1}

\makeatletter
\@ifpackageloaded{caption}{}{\usepackage{caption}}
\AtBeginDocument{%
\ifdefined\contentsname
  \renewcommand*\contentsname{Table of contents}
\else
  \newcommand\contentsname{Table of contents}
\fi
\ifdefined\listfigurename
  \renewcommand*\listfigurename{List of Figures}
\else
  \newcommand\listfigurename{List of Figures}
\fi
\ifdefined\listtablename
  \renewcommand*\listtablename{List of Tables}
\else
  \newcommand\listtablename{List of Tables}
\fi
\ifdefined\figurename
  \renewcommand*\figurename{Figure}
\else
  \newcommand\figurename{Figure}
\fi
\ifdefined\tablename
  \renewcommand*\tablename{Table}
\else
  \newcommand\tablename{Table}
\fi
}
\@ifpackageloaded{float}{}{\usepackage{float}}
\floatstyle{ruled}
\@ifundefined{c@chapter}{\newfloat{codelisting}{h}{lop}}{\newfloat{codelisting}{h}{lop}[chapter]}
\floatname{codelisting}{Listing}
\newcommand*\listoflistings{\listof{codelisting}{List of Listings}}
\makeatother
\makeatletter
\makeatother
\makeatletter
\@ifpackageloaded{caption}{}{\usepackage{caption}}
\@ifpackageloaded{subcaption}{}{\usepackage{subcaption}}
\makeatother

\ifLuaTeX
  \usepackage{selnolig}  % disable illegal ligatures
\fi
\usepackage{bookmark}

\IfFileExists{xurl.sty}{\usepackage{xurl}}{} % add URL line breaks if available
\urlstyle{same} % disable monospaced font for URLs
\hypersetup{
  pdftitle={Hierarchical Bayesian Modeling of Antibody Kinetics: Extensions and Refinements},
  pdfauthor={Kwan Ho Lee},
  colorlinks=true,
  linkcolor={blue},
  filecolor={Maroon},
  citecolor={Blue},
  urlcolor={Blue},
  pdfcreator={LaTeX via pandoc}}


\title{Hierarchical Bayesian Modeling of Antibody Kinetics: Extensions
and Refinements}
\author{Kwan Ho Lee}
\date{2025-09-03}

\begin{document}
\maketitle


\section{Overview}\label{overview}

\begin{itemize}
\tightlist
\item
  Incorporates feedback from Dr.~Morrison, Dr.~Aiemjoy, and lab
  discussion
\item
  Focus exclusively on (Teunis and Eijkeren 2016) two-phase within-host
  model
\item
  Clarifies full hierarchical Bayesian modeling structure
\item
  Explicitly distinguishes between priors, hyperpriors, transformations
\item
  Reorders: \textbf{Start from observation model → build upward}
\end{itemize}

\begin{center}\rule{0.5\linewidth}{0.5pt}\end{center}

\section{Big Picture: What Are We
Modeling?}\label{big-picture-what-are-we-modeling}

We are modeling \textbf{how antibody levels change over time} in
response to infection, using multiple individuals and multiple
biomarkers (antigen-isotype combinations, (j = 1, 2, \(\dots\), 10)).

Goals:

\begin{itemize}
\tightlist
\item
  Understand the \textbf{average pattern} for each biomarker
\item
  Allow for \textbf{individual-level variation}
\item
  \textbf{Share information} across individuals to improve inference
\end{itemize}

This motivates using a \textbf{hierarchical Bayesian model}.

\begin{center}\rule{0.5\linewidth}{0.5pt}\end{center}

\section{Step 1: Observation Model (Data
Level)}\label{step-1-observation-model-data-level}

Observed (log-transformed) antibody levels:

\begin{equation}\phantomsection\label{eq-1}{
\log(y_{\text{obs},ij}) \sim \mathcal{N}(\mu_{\log y,ij}, \tau_j^{-1})
}\end{equation}

Where:

\begin{itemize}
\tightlist
\item
  \(y_{\text{obs},ij}\): Observed antibody level for subject \(i\) and
  biomarker \(j\)
\item
  \(\mu_{\log y,ij}\) is the \textbf{expected log antibody level},
  computed from the two-phase model using subject-level parameters
  \(\theta_{ij}\).
\item
  \(\theta_{ij}\): Subject-level latent parameters (e.g.,
  \(y_0, \alpha, \rho\)) used to define the predicted antibody curve
\item
  \(\tau_j\): Measurement precision (inverse of variance) specific to
  biomarker \(j\)
\end{itemize}

The expression above corresponds to line 54 of \texttt{model.jags}:

\begin{Shaded}
\begin{Highlighting}[numbers=left,,firstnumber=54,]
\NormalTok{     logy[subj,obs,cur\_antigen\_iso] }\SpecialCharTok{\textasciitilde{}} \FunctionTok{dnorm}\NormalTok{(mu.logy[subj,obs,cur\_antigen\_iso], prec.logy[cur\_antigen\_iso])}
\end{Highlighting}
\end{Shaded}

Measurement precision prior:

\begin{equation}\phantomsection\label{eq-2}{
\tau_j \sim \text{Gamma}(a_j, b_j)
}\end{equation}

Where:

\begin{itemize}
\tightlist
\item
  \(\tau_j\): Precision (inverse of variance) of the measurement noise
  for biomarker \(j\)
\item
  \((a_j, b_j)\): Shape and rate hyperparameters of the Gamma prior for
  precision, which control its expected value and variability
\end{itemize}

The expression above corresponds to line 75 of \texttt{model.jags}:

\begin{Shaded}
\begin{Highlighting}[numbers=left,,firstnumber=75,]
\NormalTok{  prec.logy[cur\_antigen\_iso] }\SpecialCharTok{\textasciitilde{}} \FunctionTok{dgamma}\NormalTok{(prec.logy.hyp[cur\_antigen\_iso,}\DecValTok{1}\NormalTok{], prec.logy.hyp[cur\_antigen\_iso,}\DecValTok{2}\NormalTok{])}
\end{Highlighting}
\end{Shaded}

\begin{center}\rule{0.5\linewidth}{0.5pt}\end{center}

\section{Parameter Summary}\label{parameter-summary}

\begin{longtable}[]{@{}ll@{}}
\caption{Parameter summary for antibody kinetics
model.}\label{tbl-param-summary}\tabularnewline
\toprule\noalign{}
Symbol & Description \\
\midrule\noalign{}
\endfirsthead
\toprule\noalign{}
Symbol & Description \\
\midrule\noalign{}
\endhead
\bottomrule\noalign{}
\endlastfoot
\(\mu_y\) & Antibody production rate (growth phase) \\
\(\mu_b\) & Pathogen replication rate \\
\(\gamma\) & Clearance rate (by antibodies) \\
\(\alpha\) & Antibody decay rate \\
\(\rho\) & Shape of antibody decay (power-law) \\
\(t_1\) & Time of peak response \\
\(y_1\) & Peak antibody concentration \\
\end{longtable}

\textbf{Note:} Only the first 6 are typically estimated. \(y_1\) is
derived from the ODE solution at \(t_1\).

\begin{center}\rule{0.5\linewidth}{0.5pt}\end{center}

\section{Step 2: Within-Host ODE System (Teunis and Eijkeren
2016)}\label{step-2-within-host-ode-system-teunis2016}

\begin{equation}\phantomsection\label{eq-3}{
\frac{dy}{dt} = 
\begin{cases}
\mu_y y(t), & t \leq t_1 \\
- \alpha y(t)^\rho, & t > t_1
\end{cases}
\quad \text{and} \quad
\frac{db}{dt} = \mu_b b(t) - \gamma y(t)
}\end{equation}

\begin{itemize}
\tightlist
\item
  Initial conditions: \(y(0) = y_0\), \(b(0) = b_0\)
\item
  Transition at \(t_1\): when \(b(t_1) = 0\)
\end{itemize}

\begin{center}\rule{0.5\linewidth}{0.5pt}\end{center}

\section{Step 3: Closed-Form
Solutions}\label{step-3-closed-form-solutions}

\textbf{Antibody concentration:}

\begin{itemize}
\tightlist
\item
  For \(t \leq t_1\): \begin{equation}\phantomsection\label{eq-4}{
  y(t) = y_0 e^{\mu_y t}
  }\end{equation}
\item
  For \(t \> t_1\): \begin{equation}\phantomsection\label{eq-5}{
  y(t) = y_1 \left(1 + (\rho-1)\alpha y_1^{\rho-1}(t-t_1)\right)^{-\frac{1}{\rho-1}}
  }\end{equation}
\end{itemize}

The expression above corresponds to lines 18-50 of \texttt{model.jags}:

\begin{Shaded}
\begin{Highlighting}[numbers=left,,firstnumber=18,]
\NormalTok{     mu.logy[subj, obs, cur\_antigen\_iso] }\OtherTok{\textless{}{-}} \FunctionTok{ifelse}\NormalTok{(}
        
        \CommentTok{\# \textasciigrave{}step(x)\textasciigrave{} returns 1 if x \textgreater{}= 0;}
        \CommentTok{\# here we are determining which phase of infection we are in; }
        \CommentTok{\# active or recovery;}
        \CommentTok{\# \textasciigrave{}smpl.t\textasciigrave{} is the time when the blood sample was collected, }
        \CommentTok{\# relative to estimated start of infection;}
        \CommentTok{\# so we are determining whether the current observation is after \textasciigrave{}t1\textasciigrave{} }
        \CommentTok{\# the time when the active infection ended.}
        \FunctionTok{step}\NormalTok{(t1[subj,cur\_antigen\_iso] }\SpecialCharTok{{-}}\NormalTok{ smpl.t[subj,obs]), }
        
        \DocumentationTok{\#\# active infection period:}
        \CommentTok{\# this is equation 15, case t \textless{}= t\_1, but on a logarithmic scale}
        \FunctionTok{log}\NormalTok{(y0[subj,cur\_antigen\_iso]) }\SpecialCharTok{+}\NormalTok{ (beta[subj,cur\_antigen\_iso] }\SpecialCharTok{*}\NormalTok{ smpl.t[subj,obs]),}
        
        \DocumentationTok{\#\# recovery period:}
        \CommentTok{\# this is equation 15, case t \textgreater{} t\_1}
        \DecValTok{1} \SpecialCharTok{/}\NormalTok{ (}\DecValTok{1} \SpecialCharTok{{-}}\NormalTok{ shape[subj,cur\_antigen\_iso]) }\SpecialCharTok{*}
           \FunctionTok{log}\NormalTok{(}
              \CommentTok{\# this is \textasciigrave{}log\{y\_1\^{}(1{-}r)\}\textasciigrave{}; }
              \CommentTok{\# the exponent cancels out with the factor outside the log}
\NormalTok{              y1[subj, cur\_antigen\_iso]}\SpecialCharTok{\^{}}\NormalTok{(}\DecValTok{1} \SpecialCharTok{{-}}\NormalTok{ shape[subj, cur\_antigen\_iso]) }\SpecialCharTok{{-}} 
                 
               \CommentTok{\# this is (1{-}r); not sure why switched from paper  }
\NormalTok{              (}\DecValTok{1} \SpecialCharTok{{-}}\NormalTok{ shape[subj,cur\_antigen\_iso]) }\SpecialCharTok{*}
                
                  \CommentTok{\# (there\textquotesingle{}s no missing y1\^{}(r{-}1) term here; the math checks out)}
                 
                 \CommentTok{\# alpha is \textasciigrave{}nu\textasciigrave{} in Teunis 2016; the "decay rate" parameter}
\NormalTok{                alpha[subj,cur\_antigen\_iso] }\SpecialCharTok{*}
                 
                 \CommentTok{\# this is \textasciigrave{}t {-} t\_1\textasciigrave{}}
\NormalTok{                 (smpl.t[subj,obs] }\SpecialCharTok{{-}}\NormalTok{ t1[subj,cur\_antigen\_iso])))}
\end{Highlighting}
\end{Shaded}

\textbf{Pathogen load:}

\begin{itemize}
\tightlist
\item
  For \(t \leq t_1\): \begin{equation}\phantomsection\label{eq-6}{
  b(t) = b_0 e^{\mu_b t} - \frac{\gamma y_0}{\mu_y - \mu_b} \left( e^{\mu_y t} - e^{\mu_b t} \right)
  }\end{equation}
\item
  For \(t \> t_1\): \[
  b(t) = 0
  \]
\end{itemize}

\begin{center}\rule{0.5\linewidth}{0.5pt}\end{center}

\section{Step 4: Derived Quantities}\label{step-4-derived-quantities}

\begin{itemize}
\item
  \textbf{Clearance Time} \(t_1\):

  \begin{equation}\phantomsection\label{eq-7}{
  t_1 = \frac{1}{\mu_y - \mu_b} \log\left(1 + \frac{(\mu_y - \mu_b) b_0}{\gamma y_0}\right)
  }\end{equation}
\end{itemize}

The expression above is indirectly represented by lines 8-12 of
\texttt{model.jags}:

\begin{Shaded}
\begin{Highlighting}[numbers=left,,firstnumber=8,]
\NormalTok{     beta[subj, cur\_antigen\_iso] }\OtherTok{\textless{}{-}} 
       \FunctionTok{log}\NormalTok{(}
\NormalTok{         y1[subj,cur\_antigen\_iso] }\SpecialCharTok{/}\NormalTok{ y0[subj,cur\_antigen\_iso]}
\NormalTok{         ) }\SpecialCharTok{/} 
\NormalTok{       t1[subj,cur\_antigen\_iso]}
\end{Highlighting}
\end{Shaded}

\begin{itemize}
\item
  \textbf{Peak Antibody Level} \(y_1\):

  \begin{equation}\phantomsection\label{eq-8}{
  y_1 = y_0 e^{\mu_y t_1}
  }\end{equation}
\end{itemize}

The expression above corresponds to line 59 of \texttt{model.jags}:

\begin{Shaded}
\begin{Highlighting}[numbers=left,,firstnumber=59,]
\NormalTok{   y1[subj,cur\_antigen\_iso]    }\OtherTok{\textless{}{-}}\NormalTok{ y0[subj,cur\_antigen\_iso] }\SpecialCharTok{+} \FunctionTok{exp}\NormalTok{(par[subj,cur\_antigen\_iso,}\DecValTok{2}\NormalTok{]) }\CommentTok{\# par[,,2] must be log(y1{-}y0)}
\end{Highlighting}
\end{Shaded}

\textbf{Important}: \(t_1\) and \(y_1\) are \textbf{derived}, not fit
parameters.

\begin{center}\rule{0.5\linewidth}{0.5pt}\end{center}

\section{Full Parameter Model (7
Parameters)}\label{full-parameter-model-7-parameters}

\textbf{Subject-level parameters} for each subject\(i\) and biomarker
\(j\):

\begin{equation}\phantomsection\label{eq-9}{
\theta_{ij} \sim \mathcal{N}(\mu_j,\, \Sigma_j), \quad \theta_{ij} =
\begin{bmatrix}
y_{0,ij} \\
b_{0,ij} \\
\mu_{b,ij} \\
\mu_{y,ij} \\
\gamma_{ij} \\
\alpha_{ij} \\
\rho_{ij}
\end{bmatrix}
}\end{equation}

\begin{itemize}
\tightlist
\item
  These 7 parameters represent the \textbf{full biological model}
  (antibody + pathogen dynamics)
\end{itemize}

\begin{center}\rule{0.5\linewidth}{0.5pt}\end{center}

\section{From Full 7 Parameters to 5 Latent
Parameters}\label{from-full-7-parameters-to-5-latent-parameters}

\begin{itemize}
\tightlist
\item
  Although the model estimates 7 parameters, for modeling antibody
  kinetics \(y(t)\), we focus on \textbf{5-parameter subset}:
\end{itemize}

\[y_0,\ \ t_1 (\text{derived}),\ \  y_1 (\text{derived}),\ \  \alpha,\ \  \rho\]

\begin{itemize}
\tightlist
\item
  These 5 parameters are \textbf{log-transformed} into the latent
  parameters \(\theta\_{ij}\) used for modeling.
\end{itemize}

\begin{center}\rule{0.5\linewidth}{0.5pt}\end{center}

\section{Core Parameters Used for Curve
Drawing}\label{core-parameters-used-for-curve-drawing}

Although the full model estimates \textbf{7 parameters}, only \textbf{5
key parameters} required to draw antibody curves:

\begin{itemize}
\tightlist
\item
  \(y_0\): initial antibody level
\item
  \(t_1\): time of peak antibody response (derived)
\item
  \(y_1\): peak antibody level (derived)
\item
  \(\alpha\): decay rate
\item
  \(\rho\): shape of decay
\end{itemize}

Note: \(t_1\) and \(y_1\) are \textbf{derived from the full model} -
These 5 are sufficient for prediction and plotting

\begin{center}\rule{0.5\linewidth}{0.5pt}\end{center}

\section{Step 5: Subject-Level Parameters (Latent
Version)}\label{step-5-subject-level-parameters-latent-version}

Each subject \(i\) and biomarker \(j\) has latent parameters:

\begin{equation}\phantomsection\label{eq-10}{
\theta_{ij} =
\begin{bmatrix}
\log(y_{0,ij}) \\
\log(y_{1,ij} - y_{0,ij}) \\
\log(t_{1,ij}) \\
\log(\alpha_{ij}) \\
\log(\rho_{ij} - 1)
\end{bmatrix}
}\end{equation}

Distribution:

\[
\theta_{ij} \sim \mathcal{N}(\mu_j, \Sigma_j)
\]

The expression above reflects the prior distribution specified on line
66 of \texttt{model.jags}:

\begin{Shaded}
\begin{Highlighting}[numbers=left,,firstnumber=66,]
\NormalTok{   par[subj, cur\_antigen\_iso, }\DecValTok{1}\SpecialCharTok{:}\NormalTok{n\_params] }\SpecialCharTok{\textasciitilde{}} \FunctionTok{dmnorm}\NormalTok{(mu.par[cur\_antigen\_iso,], prec.par[cur\_antigen\_iso,,])}
\end{Highlighting}
\end{Shaded}

\begin{center}\rule{0.5\linewidth}{0.5pt}\end{center}

\section{Step 6: Parameter Transformations (log scale
priors)}\label{step-6-parameter-transformations-log-scale-priors}

JAGS implements latent parameters (par) as:

\begin{longtable}[]{@{}ll@{}}
\caption{Log-Scale Transformations of Antibody Model Parameters in
JAGS.}\label{tbl-transformation}\tabularnewline
\toprule\noalign{}
Model Parameter & Transformation in JAGS \\
\midrule\noalign{}
\endfirsthead
\toprule\noalign{}
Model Parameter & Transformation in JAGS \\
\midrule\noalign{}
\endhead
\bottomrule\noalign{}
\endlastfoot
\(y_0\) & \(\exp(\text{par}_1)\) \\
\(y_1\) & \(y_0 + \exp(\text{par}_2)\) \\
\(t_1\) & \(\exp(\text{par}_3)\) \\
\(\alpha\) & \(\exp(\text{par}_4)\) \\
\(\rho\) & \(\exp(\text{par}_5) + 1\) \\
\end{longtable}

The table above corresponds to lines 58-62 of \texttt{model.jags}:

\begin{Shaded}
\begin{Highlighting}[numbers=left,,firstnumber=58,]
\NormalTok{   y0[subj,cur\_antigen\_iso]    }\OtherTok{\textless{}{-}} \FunctionTok{exp}\NormalTok{(par[subj,cur\_antigen\_iso,}\DecValTok{1}\NormalTok{])}
\NormalTok{   y1[subj,cur\_antigen\_iso]    }\OtherTok{\textless{}{-}}\NormalTok{ y0[subj,cur\_antigen\_iso] }\SpecialCharTok{+} \FunctionTok{exp}\NormalTok{(par[subj,cur\_antigen\_iso,}\DecValTok{2}\NormalTok{]) }\CommentTok{\# par[,,2] must be log(y1{-}y0)}
\NormalTok{   t1[subj,cur\_antigen\_iso]    }\OtherTok{\textless{}{-}} \FunctionTok{exp}\NormalTok{(par[subj,cur\_antigen\_iso,}\DecValTok{3}\NormalTok{])}
\NormalTok{   alpha[subj,cur\_antigen\_iso] }\OtherTok{\textless{}{-}} \FunctionTok{exp}\NormalTok{(par[subj,cur\_antigen\_iso,}\DecValTok{4}\NormalTok{]) }\CommentTok{\# \textasciigrave{}nu\textasciigrave{} in the paper}
\NormalTok{   shape[subj,cur\_antigen\_iso] }\OtherTok{\textless{}{-}} \FunctionTok{exp}\NormalTok{(par[subj,cur\_antigen\_iso,}\DecValTok{5}\NormalTok{]) }\SpecialCharTok{+} \DecValTok{1} \CommentTok{\# \textasciigrave{}r\textasciigrave{} in the paper}
\end{Highlighting}
\end{Shaded}

All priors are thus applied on \textbf{log scale} (or log-minus-one for
\(\rho\)).

\begin{center}\rule{0.5\linewidth}{0.5pt}\end{center}

\section{Step 7: Population-Level Parameters
(Priors)}\label{step-7-population-level-parameters-priors}

The biomarker-specific mean vector \(\mu_j\) has a \textbf{hyperprior} :

\begin{equation}\phantomsection\label{eq-11}{
\mu_j \sim \mathcal{N}(\mu_{\text{hyp},j}, \Omega_{\text{hyp},j})
}\end{equation}

Where:

\begin{itemize}
\tightlist
\item
  \(\mu_{\text{hyp},j}\) : \textbf{prior mean} for the population-level
  parameters\\
\item
  \(\Omega_{\text{hyp},j}\) : \textbf{prior covariance} encoding
  uncertainty about \(\mu_j\) (e.g., \(100 \cdot I_7\) for weakly
  informative prior)
\end{itemize}

The expression above corresponds to line 73 of \texttt{model.jags}:

\begin{Shaded}
\begin{Highlighting}[numbers=left,,firstnumber=73,]
\NormalTok{  mu.par[cur\_antigen\_iso, }\DecValTok{1}\SpecialCharTok{:}\NormalTok{n\_params] }\SpecialCharTok{\textasciitilde{}} \FunctionTok{dmnorm}\NormalTok{(mu.hyp[cur\_antigen\_iso,], prec.hyp[cur\_antigen\_iso,,])}
\end{Highlighting}
\end{Shaded}

\textbf{Clarification:}

\begin{itemize}
\item
  \(\mu_{\text{hyp},j}\) defines the \textbf{center of a distribution,
  not} a single point guess.
\item
  In Bayesian modeling, \textbf{priors and hyperpriors are
  distributions} over unknown quantities, capturing full uncertainty.
\end{itemize}

\begin{center}\rule{0.5\linewidth}{0.5pt}\end{center}

\section{Step 8: Prior on Covariance
Matrices}\label{step-8-prior-on-covariance-matrices}

We also don't know how much individual parameters vary. So we assign a
\textbf{Wishart prior} to the \textbf{inverse} covariance matrix:

\begin{equation}\phantomsection\label{eq-12}{
\Sigma_j^{-1} \sim \mathcal{W}(\Omega_j, \nu_j)
}\end{equation}

\begin{itemize}
\tightlist
\item
  \(\Omega_j\) : prior scale matrix (small variance across parameters,
  often \(0.1 \cdot I_7\))
\item
  \(\nu_j\) : degrees of freedom
\end{itemize}

The expression above corresponds to line 74 of \texttt{model.jags}:

\begin{Shaded}
\begin{Highlighting}[numbers=left,,firstnumber=74,]
\NormalTok{  prec.par[cur\_antigen\_iso, }\DecValTok{1}\SpecialCharTok{:}\NormalTok{n\_params, }\DecValTok{1}\SpecialCharTok{:}\NormalTok{n\_params] }\SpecialCharTok{\textasciitilde{}} \FunctionTok{dwish}\NormalTok{(omega[cur\_antigen\_iso,,], wishdf[cur\_antigen\_iso])}
\end{Highlighting}
\end{Shaded}

Higher \(\nu_j\) \(\rightarrow\) more informative prior (stronger
prior).

Lower \(\nu_j\) \(\rightarrow\) more weakly informative (broader prior
or weaker prior).

This tells the model how much we expect individuals to vary from the
average for biomarker \(j\).

\begin{center}\rule{0.5\linewidth}{0.5pt}\end{center}

\section{Putting It All Together}\label{putting-it-all-together}

The model is built hierarchically across five conceptual levels:

\begin{enumerate}
\def\labelenumi{\arabic{enumi}.}
\tightlist
\item
  \textbf{Observed data:} noisy log antibody concentrations from serum
  samples
\item
  \textbf{Latent individual parameters:} hidden antibody dynamics
  \(\theta_{ij}\) for each subject-biomarker pair
\item
  \textbf{Population-level means:} average antibody parameters for each
  biomarker
\item
  \textbf{Hyperpriors on means:} our belief about the likely range of
  biomarker-specific population means
\item
  \textbf{Priors on variability:} our belief about how much individual
  parameters vary around the population mean
\end{enumerate}

This structure allows us to account for uncertainty at every level,
while borrowing strength across subjects and biomarkers.

\begin{center}\rule{0.5\linewidth}{0.5pt}\end{center}

\section{Summary of the Hierarchy}\label{summary-of-the-hierarchy}

\begin{enumerate}
\def\labelenumi{\arabic{enumi}.}
\item
  \textbf{Top Level}:

  \begin{itemize}
  \tightlist
  \item
    For each biomarker \(j\), the true mean antibody trajectory
    parameters \(\mu_j\) come from a prior:

    \begin{itemize}
    \tightlist
    \item
      \(\mu_j \sim \mathcal{N}(\mu_{\text{hyp},j}, \Omega_{\text{hyp},j})\)
    \end{itemize}
  \end{itemize}
\item
  \textbf{Middle Level}:

  \begin{itemize}
  \tightlist
  \item
    For each person \(i\), their parameters:

    \begin{itemize}
    \tightlist
    \item
      \(\theta_{ij} \sim \mathcal{N}(\mu_j, \Sigma_j)\)
    \end{itemize}
  \end{itemize}
\item
  \textbf{Bottom Level}:

  \begin{itemize}
  \tightlist
  \item
    Their actual observed antibody levels are noisy measurements of
    predictions from \(\theta_{ij}\):

    \begin{itemize}
    \tightlist
    \item
      \(\log(y_{\text{obs},ij}) \sim \mathcal{N}(\mu_{\log y,ij}, \tau_j^{-1})\)
    \end{itemize}
  \end{itemize}
\end{enumerate}

Where:

\begin{itemize}
\tightlist
\item
  \(\mu_{\log y,ij}\) is the \textbf{expected log antibody level},
  computed from the two-phase model using subject-level parameters
  \(\theta_{ij}\).
\item
  Predictions use \(\theta_{ij}\) to compute \(\mu_{\log y,ij}\), which
  is then compared to the observed log antibody data.
\end{itemize}

\begin{center}\rule{0.5\linewidth}{0.5pt}\end{center}

\section{Clarification: How Bottom Level Depends on Middle
Level}\label{clarification-how-bottom-level-depends-on-middle-level}

We know the following facts:

\begin{enumerate}
\def\labelenumi{\arabic{enumi}.}
\tightlist
\item
  \(\theta_{ij}\) are the \textbf{subject-level latent parameters} (like
  \(y_0, b_0, \mu\_b, \mu\_y, \gamma, \alpha, \rho\)).
\item
  From \(\theta_{ij}\), we calculate the expected \textbf{log antibody
  level} \(\mu_{\log y,ij}\) using the ODE-based two-phase model.
\item
  The \textbf{observed log-antibody} \(\log(y_{\text{obs},ij})\) is
  modeled as a \textbf{noisy version} of \(\mu_{\log y,ij}\).
\item
  \(\tau_j\) is the precision (measurement noise precision for biomarker
  \(j\)).
\end{enumerate}

Thus, at the \textbf{Bottom Level}, we model:

\[
\log(y_{\text{obs},ij}) \sim \mathcal{N}(\mu_{\log y,ij}, \tau_j^{-1})
\]

Here:

\begin{itemize}
\tightlist
\item
  The \textbf{mean} is \(\mu_{\log y,ij}\) --- derived from the
  \textbf{ODE solution} using \(\theta_{ij}\).
\item
  The \textbf{variance} is \(\tau_j^{-1}\) --- shared across individuals
  for a given biomarker.
\end{itemize}

Summary:

\begin{itemize}
\tightlist
\item
  Observations depend indirectly on latent parameters \(\theta_{ij}\)
  via the predicted log antibody levels \(\mu_{\log y,ij}\).
\end{itemize}

\begin{center}\rule{0.5\linewidth}{0.5pt}\end{center}

\section{Summary Mapping of Notation}\label{summary-mapping-of-notation}

\begin{longtable}[]{@{}
  >{\raggedright\arraybackslash}p{(\columnwidth - 4\tabcolsep) * \real{0.2500}}
  >{\raggedright\arraybackslash}p{(\columnwidth - 4\tabcolsep) * \real{0.4444}}
  >{\raggedright\arraybackslash}p{(\columnwidth - 4\tabcolsep) * \real{0.3056}}@{}}
\toprule\noalign{}
\begin{minipage}[b]{\linewidth}\raggedright
Symbol
\end{minipage} & \begin{minipage}[b]{\linewidth}\raggedright
Meaning
\end{minipage} & \begin{minipage}[b]{\linewidth}\raggedright
JAGS Variable
\end{minipage} \\
\midrule\noalign{}
\endhead
\bottomrule\noalign{}
\endlastfoot
\(i\) & Subject index & \texttt{subj} \\
\(j\) & Antigen-isotype (biomarker) index &
\texttt{cur\_antigen\_iso} \\
\(y_{\text{obs},ij}\) & Observed antibody concentration at a timepoint &
\texttt{logy{[}subj,\ obs,\ cur\_antigen\_iso{]}} \\
\(\mu_{\log y,ij}\) & Expected log antibody level based on ODE model
using \(\theta_{ij}\) &
\texttt{mu.logy{[}subj,\ obs,\ cur\_antigen\_iso{]}} \\
\(\theta_{ij}\) & Subject-level latent parameters for modeling \(y(t)\)
& \texttt{par{[}subj,\ cur\_antigen\_iso,\ 1:n\_params{]}} \\
\(\mu_j\) & Mean vector of latent parameters across subjects for
biomarker \(j\) & \texttt{mu.par{[}cur\_antigen\_iso,\ {]}} \\
\(\Sigma_j\) & Covariance matrix of latent parameters for biomarker
\(j\) & \texttt{inverse\ of\ prec.par{[}cur\_antigen\_iso,\ ,\ {]}} \\
\(\tau_j\) & Precision (inverse variance) of measurement error for
biomarker \(j\) & \texttt{prec.logy{[}cur\_antigen\_iso{]}} \\
\((a_j, b_j)\) & Gamma prior hyperparameters for \(\tau_j\) &
\texttt{prec.logy.hyp{[}cur\_antigen\_iso,\ 1/2{]}} \\
\(\mu_{\text{hyp},j}\) & Prior mean for \(\mu_j\) &
\texttt{mu.hyp{[}cur\_antigen\_iso,\ {]}} \\
\(\Omega_{\text{hyp},j}\) & Prior precision for \(\mu_j\) &
\texttt{prec.hyp{[}cur\_antigen\_iso,\ ,\ {]}} \\
\((\Omega_j, \nu_j)\) & Wishart scale and degrees of freedom for
\(\Sigma_j^{-1}\) &
\texttt{omega{[}cur\_antigen\_iso,\ ,\ {]},\ wishdf{[}...{]}} \\
\end{longtable}

\begin{center}\rule{0.5\linewidth}{0.5pt}\end{center}

\section{Model Comparison (Teunis and Eijkeren 2016) vs.~Our
Presentation}\label{model-comparison-teunis2016-vs.-our-presentation}

\begin{longtable}[]{@{}
  >{\raggedright\arraybackslash}p{(\columnwidth - 4\tabcolsep) * \real{0.2917}}
  >{\raggedright\arraybackslash}p{(\columnwidth - 4\tabcolsep) * \real{0.3194}}
  >{\raggedright\arraybackslash}p{(\columnwidth - 4\tabcolsep) * \real{0.3889}}@{}}
\caption{Comparison of Teunis (2016) model and this presentation's model
assumptions.}\label{tbl-model-comparison}\tabularnewline
\toprule\noalign{}
\begin{minipage}[b]{\linewidth}\raggedright
Component
\end{minipage} & \begin{minipage}[b]{\linewidth}\raggedright
(Teunis and Eijkeren 2016)
\end{minipage} & \begin{minipage}[b]{\linewidth}\raggedright
This Presentation
\end{minipage} \\
\midrule\noalign{}
\endfirsthead
\toprule\noalign{}
\begin{minipage}[b]{\linewidth}\raggedright
Component
\end{minipage} & \begin{minipage}[b]{\linewidth}\raggedright
(Teunis and Eijkeren 2016)
\end{minipage} & \begin{minipage}[b]{\linewidth}\raggedright
This Presentation
\end{minipage} \\
\midrule\noalign{}
\endhead
\bottomrule\noalign{}
\endlastfoot
Pathogen ODE & \(\mu_0 b(t) - c y(t)\) & \(\mu_b b(t) - \gamma y(t)\) \\
Antibody growth ODE & \(\mu y(t)\) & \(\mu_y y(t)\) \\
Antibody decay ODE & \(-\alpha y(t)^r\) & \(-\alpha y(t)^\rho\) \\
Growth mechanism & Pathogen-driven & Self-driven \\
\end{longtable}

\phantomsection\label{refs}
\begin{CSLReferences}{1}{0}
\bibitem[\citeproctext]{ref-teunis2016}
Teunis, Peter F. M., and J. C. H. van Eijkeren. 2016. {``Linking the
Seroresponse to Infection to Within-Host Heterogeneity in Antibody
Production.''} \emph{Epidemics} 16: 33--39.
\url{https://doi.org/10.1016/j.epidem.2016.04.001}.

\end{CSLReferences}




\end{document}
