% Options for packages loaded elsewhere
\PassOptionsToPackage{unicode}{hyperref}
\PassOptionsToPackage{hyphens}{url}
%
\documentclass[
  ignorenonframetext,
]{beamer}
\usepackage{pgfpages}
\setbeamertemplate{caption}[numbered]
\setbeamertemplate{caption label separator}{: }
\setbeamercolor{caption name}{fg=normal text.fg}
\beamertemplatenavigationsymbolsempty
% Prevent slide breaks in the middle of a paragraph
\widowpenalties 1 10000
\raggedbottom
\setbeamertemplate{part page}{
  \centering
  \begin{beamercolorbox}[sep=16pt,center]{part title}
    \usebeamerfont{part title}\insertpart\par
  \end{beamercolorbox}
}
\setbeamertemplate{section page}{
  \centering
  \begin{beamercolorbox}[sep=12pt,center]{part title}
    \usebeamerfont{section title}\insertsection\par
  \end{beamercolorbox}
}
\setbeamertemplate{subsection page}{
  \centering
  \begin{beamercolorbox}[sep=8pt,center]{part title}
    \usebeamerfont{subsection title}\insertsubsection\par
  \end{beamercolorbox}
}
\AtBeginPart{
  \frame{\partpage}
}
\AtBeginSection{
  \ifbibliography
  \else
    \frame{\sectionpage}
  \fi
}
\AtBeginSubsection{
  \frame{\subsectionpage}
}

\usepackage{amsmath,amssymb}
\usepackage{iftex}
\ifPDFTeX
  \usepackage[T1]{fontenc}
  \usepackage[utf8]{inputenc}
  \usepackage{textcomp} % provide euro and other symbols
\else % if luatex or xetex
  \usepackage{unicode-math}
  \defaultfontfeatures{Scale=MatchLowercase}
  \defaultfontfeatures[\rmfamily]{Ligatures=TeX,Scale=1}
\fi
\usepackage{lmodern}
\usetheme[]{Madrid}
\ifPDFTeX\else  
    % xetex/luatex font selection
\fi
% Use upquote if available, for straight quotes in verbatim environments
\IfFileExists{upquote.sty}{\usepackage{upquote}}{}
\IfFileExists{microtype.sty}{% use microtype if available
  \usepackage[]{microtype}
  \UseMicrotypeSet[protrusion]{basicmath} % disable protrusion for tt fonts
}{}
\makeatletter
\@ifundefined{KOMAClassName}{% if non-KOMA class
  \IfFileExists{parskip.sty}{%
    \usepackage{parskip}
  }{% else
    \setlength{\parindent}{0pt}
    \setlength{\parskip}{6pt plus 2pt minus 1pt}}
}{% if KOMA class
  \KOMAoptions{parskip=half}}
\makeatother
\usepackage{xcolor}
\newif\ifbibliography
\setlength{\emergencystretch}{3em} % prevent overfull lines
\setcounter{secnumdepth}{-\maxdimen} % remove section numbering


\providecommand{\tightlist}{%
  \setlength{\itemsep}{0pt}\setlength{\parskip}{0pt}}\usepackage{longtable,booktabs,array}
\usepackage{calc} % for calculating minipage widths
\usepackage{caption}
% Make caption package work with longtable
\makeatletter
\def\fnum@table{\tablename~\thetable}
\makeatother
\usepackage{graphicx}
\makeatletter
\def\maxwidth{\ifdim\Gin@nat@width>\linewidth\linewidth\else\Gin@nat@width\fi}
\def\maxheight{\ifdim\Gin@nat@height>\textheight\textheight\else\Gin@nat@height\fi}
\makeatother
% Scale images if necessary, so that they will not overflow the page
% margins by default, and it is still possible to overwrite the defaults
% using explicit options in \includegraphics[width, height, ...]{}
\setkeys{Gin}{width=\maxwidth,height=\maxheight,keepaspectratio}
% Set default figure placement to htbp
\makeatletter
\def\fps@figure{htbp}
\makeatother

\makeatletter
\@ifpackageloaded{caption}{}{\usepackage{caption}}
\AtBeginDocument{%
\ifdefined\contentsname
  \renewcommand*\contentsname{Table of contents}
\else
  \newcommand\contentsname{Table of contents}
\fi
\ifdefined\listfigurename
  \renewcommand*\listfigurename{List of Figures}
\else
  \newcommand\listfigurename{List of Figures}
\fi
\ifdefined\listtablename
  \renewcommand*\listtablename{List of Tables}
\else
  \newcommand\listtablename{List of Tables}
\fi
\ifdefined\figurename
  \renewcommand*\figurename{Figure}
\else
  \newcommand\figurename{Figure}
\fi
\ifdefined\tablename
  \renewcommand*\tablename{Table}
\else
  \newcommand\tablename{Table}
\fi
}
\@ifpackageloaded{float}{}{\usepackage{float}}
\floatstyle{ruled}
\@ifundefined{c@chapter}{\newfloat{codelisting}{h}{lop}}{\newfloat{codelisting}{h}{lop}[chapter]}
\floatname{codelisting}{Listing}
\newcommand*\listoflistings{\listof{codelisting}{List of Listings}}
\makeatother
\makeatletter
\makeatother
\makeatletter
\@ifpackageloaded{caption}{}{\usepackage{caption}}
\@ifpackageloaded{subcaption}{}{\usepackage{subcaption}}
\makeatother

\ifLuaTeX
  \usepackage{selnolig}  % disable illegal ligatures
\fi
\usepackage{bookmark}

\IfFileExists{xurl.sty}{\usepackage{xurl}}{} % add URL line breaks if available
\urlstyle{same} % disable monospaced font for URLs
\hypersetup{
  pdftitle={Hierarchical Model for Antibody Kinetics: Revisions Based on Advisor Feedback},
  pdfauthor={Kwan Ho Lee},
  hidelinks,
  pdfcreator={LaTeX via pandoc}}


\title{Hierarchical Model for Antibody Kinetics: Revisions Based on
Advisor Feedback}
\author{Kwan Ho Lee}
\date{2025-04-14}
\institute{UC Davis}

\begin{document}
\frame{\titlepage}


\begin{frame}{Overview}
\phantomsection\label{overview}
\begin{itemize}
\tightlist
\item
  Incorporates feedback from Dr.~Morrison
\item
  Aligns with Teunis et al.~(2016, 2023) formulations
\item
  Clarifies model parameter roles and their interpretation
\item
  Assumes block-diagonal covariance structure across biomarkers
\end{itemize}
\end{frame}

\begin{frame}{Full Model Structure}
\phantomsection\label{full-model-structure}
\textbf{Two-phase within-host antibody kinetics:}

\[
\frac{dy}{dt} = 
\begin{cases}
\mu_1 b(t), & t < t_1 \\
- \alpha y(t)^r, & t \ge t_1
\end{cases}
\quad \text{with } 
\frac{db}{dt} = \mu_0 b(t) - c y(t) b(t)
\]

\textbf{Initial conditions:} \(y(0) = y_0\), \(b(0) = b_0\)\\
\textbf{Key transition:} \(t_1\) is the time when \(b(t_1) = 0\)\\
\textbf{Derived quantity:} \(y_1 = y(t_1)\)
\end{frame}

\begin{frame}{Definition of Model Quantities}
\phantomsection\label{definition-of-model-quantities}
\textbf{Parameters used in the dynamic model:}

\begin{itemize}
\tightlist
\item
  \(\mu_0\): Pathogen growth rate\\
\item
  \(\mu_1\): Antibody production rate (driven by pathogen)\\
\item
  \(c\): Clearance rate --- how effectively antibodies eliminate
  pathogen\\
\item
  \(\alpha\): Antibody decay rate (governs speed of waning)\\
\item
  \(r\): Shape of antibody decay (nonlinear power)\\
\item
  \(y_0\): Initial antibody concentration at \(t = 0\)\\
\item
  \(b_0\): Initial pathogen concentration at \(t = 0\)\\
\item
  \(y_1 = y(t_1)\): Peak antibody level --- computed at time of pathogen
  clearance
\end{itemize}

\textbf{Note:} Only the first 7 are estimated. \(y_1\) is derived from
the ODE solution.
\end{frame}

\begin{frame}{Model Comparison: 2016 vs Our Formulation}
\phantomsection\label{model-comparison-2016-vs-our-formulation}
\begin{longtable}[]{@{}lll@{}}
\toprule\noalign{}
\textbf{Component} & \textbf{Teunis (2016)} & \textbf{Our Model} \\
\midrule\noalign{}
\endhead
Pathogen ODE & \(\mu_0 b(t) - c y(t)\) & \(\mu_0 b(t) - c y(t) b(t)\) \\
Antibody ODE (pre-\(t_1\)) & \(\mu y(t)\) & \(\mu_1 b(t)\) \\
Antibody ODE (post-\(t_1\)) & \(-\alpha y(t)^r\) & Same \\
Antibody growth type & Exponential & Pathogen-driven \\
Antibody rate name & \(\mu\) & \(\mu_1\) \\
\(t_1\) formula & Uses \(\mu\) & Uses \(\mu_1\) \\
\bottomrule\noalign{}
\end{longtable}

\textbf{Note:}

\begin{itemize}
\tightlist
\item
  Antibody production depends on pathogen presence (\(b(t)\)), not
  constant exponential growth\\
\item
  Pathogen clearance is proportional to both antibody and pathogen
  levels (\(c\,y(t)\,b(t)\))
\end{itemize}
\end{frame}

\begin{frame}{Hierarchical Priors -- Subject-Level and Means}
\phantomsection\label{hierarchical-priors-subject-level-and-means}
\textbf{Subject-level parameters:}

\[
\boldsymbol{\theta}_{ij} 
\sim 
\mathcal{N}(\boldsymbol{\mu}_j,\,\Sigma_j),\quad
\boldsymbol{\theta}_{ij} = 
\begin{bmatrix}
y_{0,ij} \\
b_{0,ij} \\
\mu_{0,ij} \\
\mu_{1,ij} \\
c_{ij} \\
\alpha_{ij} \\
r_{ij}
\end{bmatrix}
\]

\textbf{Hyperparameters -- Means:}

\begin{itemize}
\tightlist
\item
  \(\boldsymbol{\mu}_j\): population-level mean vector for biomarker
  \(j\)\\
\item
  Prior on \(\boldsymbol{\mu}_j\):
\end{itemize}

\[
\boldsymbol{\mu}_j \sim \mathcal{N}(\boldsymbol{\mu}_{\mathrm{hyp},j},\,\Omega_{\mathrm{hyp},j})
\]

\begin{itemize}
\tightlist
\item
  \(\boldsymbol{\mu}_{\mathrm{hyp},j}\) and \(\Omega_{\mathrm{hyp},j}\)
  are fixed or weakly informative
\end{itemize}
\end{frame}

\begin{frame}{Clarifying Parameter Roles}
\phantomsection\label{clarifying-parameter-roles}
\textbf{Why the confusion about number of parameters?}

\begin{itemize}
\tightlist
\item
  The dynamic model contains 8 named parameters:
\end{itemize}

\[
\mu_0, \mu_1, c, \alpha, r, y_0, b_0, y_1
\]

\begin{itemize}
\tightlist
\item
  But only 7 are estimated --- the 8th (\(y_1\)) is computed.
\item
  Let's break this down carefully.
\end{itemize}
\end{frame}

\begin{frame}{Classification of Parameters}
\phantomsection\label{classification-of-parameters}
\textbf{Estimated Parameters (7 total):}

\begin{itemize}
\tightlist
\item
  \textbf{Core model parameters (5):}
\end{itemize}

\[
\mu_0,\ \mu_1,\ c,\ \alpha,\ r
\]

\begin{itemize}
\tightlist
\item
  \textbf{Initial conditions (2):}
\end{itemize}

\[
y_0,\ b_0
\]

\textbf{Derived Quantity (not estimated):}

\begin{itemize}
\tightlist
\item
  \(y_1\): peak antibody level computed as \(y(t_1)\)
\end{itemize}
\end{frame}

\begin{frame}{Time of Pathogen Clearance: \(t_1\)}
\phantomsection\label{time-of-pathogen-clearance-t_1}
\textbf{Definition:} \(t_1\) is the time at which the pathogen is
cleared, i.e., \(b(t_1) = 0\)

\textbf{Analytic expression (Teunis et al., 2016):}

\[
t_1 = \frac{1}{\mu_1 - \mu_0} \log\left(1 + \frac{(\mu_1 - \mu_0)\,b_0}{c\,y_0} \right)
\]

\textbf{Key observations:}

\begin{itemize}
\tightlist
\item
  \(t_1\) depends on \(\mu_0\), \(\mu_1\), \(b_0\), \(y_0\), and \(c\)
\item
  Used to determine \(y_1 = y(t_1)\) by solving the antibody ODE up to
  this point
\item
  Not treated as an estimated parameter --- it is computed from model
  inputs
\end{itemize}
\end{frame}

\begin{frame}{Why It's a Seven-Parameter Model}
\phantomsection\label{why-its-a-seven-parameter-model}
\begin{itemize}
\tightlist
\item
  Our model estimates 7 parameters:

  \begin{itemize}
  \tightlist
  \item
    \textbf{5 core biological parameters:}
    \(\mu_0,\ \mu_1,\ c,\ \alpha,\ r\)
  \item
    \textbf{2 initial conditions:} \(y_0,\ b_0\)
  \end{itemize}
\item
  But we often talk about an eighth quantity, \(y_1\), the highest level
  of antibody.
\item
  So why isn't \(y_1\) counted as a parameter?
\end{itemize}
\end{frame}

\begin{frame}{Why \(y_1\) Is Not Fit Directly}
\phantomsection\label{why-y_1-is-not-fit-directly}
\begin{itemize}
\tightlist
\item
  \(y_1\) is the antibody level at the time the pathogen is cleared:
\end{itemize}

\[
y_1 = y(t_1) \quad \text{where } b(t_1) = 0
\]

\begin{itemize}
\tightlist
\item
  It is not an ``input'' to the model --- we don't estimate it with
  MCMC.
\item
  Instead, we \textbf{calculate it from the model}:

  \begin{itemize}
  \tightlist
  \item
    We estimate parameters like \(\mu_1\), \(y_0\), \(b_0\)\ldots{}
  \item
    Then we solve the ODEs to find \(t_1\) and compute \(y(t_1)\)
  \end{itemize}
\item
  In other words: \(y_1\) is a \textbf{derived output}, not a parameter
  being fit.
\end{itemize}
\end{frame}

\begin{frame}{How \(y_1\) Is Computed}
\phantomsection\label{how-y_1-is-computed}
\begin{itemize}
\tightlist
\item
  \(y_1\) is computed by solving the coupled ODE system:
\end{itemize}

\[
\frac{dy}{dt} = \mu_1 b(t), \quad \frac{db}{dt} = \mu_0 b(t) - c y(t) b(t)
\]

\begin{itemize}
\tightlist
\item
  The solution is evaluated at \(t = t_1\) (pathogen clearance point).
\item
  Therefore:
\end{itemize}

\[
y_1 = y(t_1;\ \mu_1,\ y_0,\ b_0,\ \mu_0,\ c)
\]
\end{frame}

\begin{frame}{Recap: What We Estimate}
\phantomsection\label{recap-what-we-estimate}
\textbf{Seven model parameters:}

\begin{itemize}
\tightlist
\item
  \(\mu_0,\ \mu_1,\ c,\ \alpha,\ r\) (biological process)
\item
  \(y_0,\ b_0\) (initial state)
\end{itemize}

\textbf{Derived quantity:}

\begin{itemize}
\tightlist
\item
  \(y_1 = y(t_1)\), not directly estimated
\end{itemize}
\end{frame}

\begin{frame}{Hierarchical Bayesian Structure}
\phantomsection\label{hierarchical-bayesian-structure}
\textbf{Individual parameters:}

\[
\theta_{ij} = \begin{bmatrix}
y_{0,ij} \\
b_{0,ij} \\
\mu_{0,ij} \\
\mu_{1,ij} \\
c_{ij} \\
\alpha_{ij} \\
r_{ij}
\end{bmatrix} 
\sim \mathcal{N}(\mu_j, \Sigma_j)
\]

\textbf{Hyperparameters:}

\begin{itemize}
\tightlist
\item
  \(\mu_j\): population-level means (per biomarker \(j\))
\item
  \(\Sigma_j\): \(7 \times 7\) covariance matrix over parameters
\end{itemize}
\end{frame}

\begin{frame}{Subject-Level Parameters: \(\boldsymbol{\theta}_{ij}\)}
\phantomsection\label{subject-level-parameters-boldsymboltheta_ij}
\[
\boldsymbol{\theta}_{ij} \sim \mathcal{N}(\boldsymbol{\mu}_j,\,\Sigma_j),
\quad \boldsymbol{\theta}_{ij} \in \mathbb{R}^7
\]

\textbf{Where:}

\[
\boldsymbol{\theta}_{ij} = 
\begin{bmatrix}
y_{0,ij} \\
b_{0,ij} \\
\mu_{0,ij} \\
\mu_{1,ij} \\
c_{ij} \\
\alpha_{ij} \\
r_{ij}
\end{bmatrix},
\quad \Sigma_j \in \mathbb{R}^{7 \times 7}
\]

Each subject \(i\) has a unique 7-parameter vector per biomarker \(j\),
capturing individual-level variation in dynamics.
\end{frame}

\begin{frame}{Hyperparameters: Priors on Population Means}
\phantomsection\label{hyperparameters-priors-on-population-means}
\textbf{Population-level means:}

\[
\boldsymbol{\mu}_j \sim \mathcal{N}(\boldsymbol{\mu}_{\mathrm{hyp},j}, \Omega_{\mathrm{hyp},j})
\]

\textbf{Interpretation:}

\begin{itemize}
\tightlist
\item
  \(\boldsymbol{\mu}_j\): average parameter vector for biomarker \(j\)
\item
  \(\boldsymbol{\mu}_{\mathrm{hyp},j}\): prior guess (e.g., vector of
  zeros)
\item
  \(\Omega_{\mathrm{hyp},j}\): covariance matrix encoding uncertainty
\end{itemize}

\textbf{Example:}

\[
\boldsymbol{\mu}_{\mathrm{hyp},j} = 0, \quad \Omega_{\mathrm{hyp},j} = 100 \cdot I_7
\]
\end{frame}

\begin{frame}{Hyperparameters: Priors on Covariance}
\phantomsection\label{hyperparameters-priors-on-covariance}
\textbf{Covariance across parameters:}

\[
\Sigma_j^{-1} \sim \mathcal{W}(\Omega_j, \nu_j)
\]

\begin{itemize}
\tightlist
\item
  \(\Sigma_j\): variability/covariance in subject-level parameters
\item
  \(\Omega_j\): prior scale matrix
\item
  \(\nu_j\): degrees of freedom
\end{itemize}

\textbf{Example:}

\[
\Omega_j = 0.1 \cdot I_7, \quad \nu_j = 8
\]
\end{frame}

\begin{frame}{Measurement Error and Precision Priors}
\phantomsection\label{measurement-error-and-precision-priors}
\textbf{Observed antibody levels:}

\[
\log(y_{\text{obs},ij}) \sim \mathcal{N}(\log(y_{\text{pred},ij}), \tau_j^{-1})
\]

\textbf{Precision prior:}

\[
\tau_j \sim \text{Gamma}(a_j, b_j)
\]

\begin{itemize}
\tightlist
\item
  \(\tau_j\): shared measurement precision for biomarker \(j\)
\item
  Gamma prior allows flexible noise modeling
\end{itemize}
\end{frame}

\begin{frame}{Matrix Algebra Computation}
\phantomsection\label{matrix-algebra-computation}
Let \(K = 7\) (parameters), \(J\) biomarkers. Then:

\[
\Theta_i =
\begin{bmatrix}
\theta_{i1} & \theta_{i2} & \cdots & \theta_{iJ}
\end{bmatrix}
\in \mathbb{R}^{K \times J}
\]

Assume:

\[
\text{vec}(\Theta_i) \sim \mathcal{N}(\text{vec}(M), \Sigma_K \otimes I_J)
\]
\end{frame}

\begin{frame}{Matrix Algebra -- Simplified Structure}
\phantomsection\label{matrix-algebra-simplified-structure}
Setup: \(\Theta_i \in \mathbb{R}^{7 \times J}\)

Model:

\[
\text{vec}(\Theta_i) \sim \mathcal{N}(\text{vec}(M), \Sigma_K \otimes I_J)
\]

\begin{itemize}
\tightlist
\item
  \(\Sigma_K\): 7×7 covariance (same across biomarkers)
\item
  \(I_J\): biomarkers assumed uncorrelated
\item
  Block-diagonal covariance
\end{itemize}
\end{frame}

\begin{frame}{Understanding \(\text{vec}(\Theta_i)\)}
\phantomsection\label{understanding-textvectheta_i}
Each \(\theta_{ij} \in \mathbb{R}^7\):

\[
\theta_{ij} =
\begin{bmatrix}
y_0 \\
b_0 \\
\mu_0 \\
\mu_1 \\
c \\
\alpha \\
r
\end{bmatrix}
\]

Flattening:

\[
\text{vec}(\Theta_i) \in \mathbb{R}^{7J \times 1}
\]
\end{frame}

\begin{frame}{Understanding \(\text{vec}(M)\)}
\phantomsection\label{understanding-textvecm}
Let \(M = [\mu_1\, \mu_2\, \cdots\, \mu_J] \in \mathbb{R}^{7 \times J}\)

Example for \(J=3\):

\[
M =
\begin{bmatrix}
\mu_{1,1} & \mu_{1,2} & \mu_{1,3} \\
\mu_{2,1} & \mu_{2,2} & \mu_{2,3} \\
\mu_{3,1} & \mu_{3,2} & \mu_{3,3} \\
\mu_{4,1} & \mu_{4,2} & \mu_{4,3} \\
\mu_{5,1} & \mu_{5,2} & \mu_{5,3} \\
\mu_{6,1} & \mu_{6,2} & \mu_{6,3} \\
\mu_{7,1} & \mu_{7,2} & \mu_{7,3}
\end{bmatrix}
\]
\end{frame}

\begin{frame}{Covariance Structure: \(\Sigma_K \otimes I_J\)}
\phantomsection\label{covariance-structure-sigma_k-otimes-i_j}
\[
\text{Cov}(\text{vec}(\Theta_i)) = \Sigma_K \otimes I_J
\]

\begin{itemize}
\tightlist
\item
  \(\Sigma_K\): parameter covariance matrix
\item
  \(I_J\): biomarker-wise independence
\item
  Kronecker product yields block-diagonal matrix
\end{itemize}
\end{frame}

\begin{frame}{Example: Kronecker Product with \(K=2\), \(J=3\)}
\phantomsection\label{example-kronecker-product-with-k2-j3}
Let:

\[
\Sigma_K =
\begin{bmatrix}
\sigma_{11} & \sigma_{12} \\
\sigma_{21} & \sigma_{22}
\end{bmatrix},\quad
I_3 =
\begin{bmatrix}
1 & 0 & 0 \\
0 & 1 & 0 \\
0 & 0 & 1
\end{bmatrix}
\]

Then:

\[
\Sigma_K \otimes I_3 \in \mathbb{R}^{6 \times 6}
\]
\end{frame}

\begin{frame}{Expanded Matrix: \(\Sigma_K \otimes I_3\)}
\phantomsection\label{expanded-matrix-sigma_k-otimes-i_3}
\[
\Sigma_K \otimes I_3 =
\begin{bmatrix}
\sigma_{11} & 0 & 0 & \sigma_{12} & 0 & 0 \\
0 & \sigma_{11} & 0 & 0 & \sigma_{12} & 0 \\
0 & 0 & \sigma_{11} & 0 & 0 & \sigma_{12} \\
\sigma_{21} & 0 & 0 & \sigma_{22} & 0 & 0 \\
0 & \sigma_{21} & 0 & 0 & \sigma_{22} & 0 \\
0 & 0 & \sigma_{21} & 0 & 0 & \sigma_{22}
\end{bmatrix}
\]
\end{frame}

\begin{frame}{Next Steps: Modeling Correlation Across Biomarkers}
\phantomsection\label{next-steps-modeling-correlation-across-biomarkers}
Current Limitation:

\begin{itemize}
\tightlist
\item
  Biomarkers assumed independent: \(I_J\)
\end{itemize}

Planned Extension:

\begin{itemize}
\tightlist
\item
  Use full covariance \(\Sigma_J\):
\end{itemize}

\[
\text{Cov}(\text{vec}(\Theta_i)) = \Sigma_K \otimes \Sigma_J
\]
\end{frame}

\begin{frame}{Extending to Correlated Biomarkers}
\phantomsection\label{extending-to-correlated-biomarkers}
Assume \(K=3\), \(J=3\)

Define:

\[
\Sigma_K =
\begin{bmatrix}
\sigma_{11} & \sigma_{12} & \sigma_{13} \\
\sigma_{21} & \sigma_{22} & \sigma_{23} \\
\sigma_{31} & \sigma_{32} & \sigma_{33}
\end{bmatrix},\quad
\Sigma_J =
\begin{bmatrix}
\tau_{11} & \tau_{12} & \tau_{13} \\
\tau_{21} & \tau_{22} & \tau_{23} \\
\tau_{31} & \tau_{32} & \tau_{33}
\end{bmatrix}
\]
\end{frame}

\begin{frame}{Kronecker Product Structure:
\(\Sigma_K \otimes \Sigma_J\)}
\phantomsection\label{kronecker-product-structure-sigma_k-otimes-sigma_j}
\[
\Sigma_K \otimes \Sigma_J =
\begin{bmatrix}
\sigma_{11}\Sigma_J & \sigma_{12}\Sigma_J & \sigma_{13}\Sigma_J \\
\sigma_{21}\Sigma_J & \sigma_{22}\Sigma_J & \sigma_{23}\Sigma_J \\
\sigma_{31}\Sigma_J & \sigma_{32}\Sigma_J & \sigma_{33}\Sigma_J
\end{bmatrix}
\]

Now biomarkers and parameters can be correlated.
\end{frame}

\begin{frame}{Expanded Form: \(\Sigma_K \otimes \Sigma_J\) (3x3)}
\phantomsection\label{expanded-form-sigma_k-otimes-sigma_j-3x3}
The \(9 \times 9\) matrix contains all combinations
\(\sigma_{ab}\tau_{cd}\)

Not block-diagonal --- includes cross-biomarker correlation
\end{frame}

\begin{frame}{Practical To-Do List (for Chapter 2)}
\phantomsection\label{practical-to-do-list-for-chapter-2}
\textbf{Model Implementation:}

\begin{itemize}
\tightlist
\item
  Define full \(\Sigma_J\) and prior:
  \(\Sigma_J^{-1} \sim \mathcal{W}(\Psi, \nu)\)\\
\item
  Implement \(\Sigma_K \otimes \Sigma_J\) in JAGS
\end{itemize}

\textbf{Simulation + Validation:}

\begin{itemize}
\tightlist
\item
  Simulate individuals with correlated biomarkers\\
\item
  Fit both block-diagonal and full-covariance models\\
\item
  Compare fit: DIC, WAIC, predictive checks
\end{itemize}
\end{frame}




\end{document}
